\documentclass{article}
\usepackage[table]{xcolor}
\usepackage{array}
\usepackage{float}
\usepackage{graphicx}
\usepackage{tikz}
\usepackage{tcolorbox}
\usepackage{amsmath}
\usepackage{enumitem}
\usepackage{tabularx} % Para tablas que se ajustan al ancho de la página
\usepackage{booktabs} % Para líneas mejoradas en la tabla

\renewcommand\thesubsection{\arabic{subsection}} % Hace que subsección use solo números (1, 2, 3, etc.)
\setcounter{secnumdepth}{2} % Asegura que las subsecciones se numerarán

\title{Aula 09 - Exercício prático Árvore Vermelho Preto}
\author{Aluno: Gian Franco Joel Condori Luna}
\date{\today}

\begin{document}

\maketitle

\section*{Exercices}
\setcounter{section}{1}
\subsection {(0,8) Insira aleatoriamente 100.000 elementos em uma árvore Vermelho e Preto.}
\begin{enumerate}[label=\alph*)]
    \item Calcule o tempo de inserção dos 100.000 elementos em cada estrutura de dados.
    \item Calcule o tempo de busca do elemento de valor 50 em cada estrutura de dados.
    Mesmo se não existir esse elemento, reporte o tempo que levou para procurá-lo.
    \item Calcule o tempo de busca do elemento de valor 50.000 em cada estrutura de
    dados. Mesmo se não existir esse elemento, reporte o tempo que levou para
    procurá-lo.
    \item Compare com os tempos de inserção e busca dos exercícios anteriores. Discuta
    sobre qual é a melhor estrutura de dados.
\end{enumerate}

\subsection*{Solução:}

(O código está no arquivo python)
\\

\begin{tabularx}{\textwidth}{|X|X|X|X|} 
  \hline
   & 
   Tempo inserção 100.000 elementos & 
   Tempo pesquisa o elemento 50 & 
   Tempo pesquisa o elemento 50.000 \\ 
  \hline
  Vetor dados aleatórios & 0.003188 s  & 0.005875 s  & 0.007157 s  \\ \hline
  Vetor ordenado + busca binária & 0.072296 s  & 0.000016 s  & 0.000011 s  \\ \hline
  Árvore binária & 1.185774 s  & 0.000015 s  & 0.000009 s  \\ \hline
  Árvore AVL & 2.764374 s  & 0.000013 s  & 0.000007 s  \\ \hline
  Árvore vermelho-preto & 0.669266 s  & 0.000067 s  & 0.000079 s  \\ \hline
\end{tabularx}
\\
\\Discussão:
\begin{itemize}
  \item \textbf{Melhor Estrutura para Inserção}: Para inserções rápidas, o vetor de dados aleatórios é o mais eficiente, mas isso pode ser enganoso se precisarmos manter os dados ordenados.

  \item \textbf{Melhor Estrutura para Busca}: O vetor ordenado com busca binária oferece o melhor desempenho em busca, mas isso exige uma ordenação prévia que pode ser custosa.
  
  \item \textbf{Melhor em Geral}: A árvore vermelho-preto parece ser a mais equilibrada, com um bom desempenho em ambas as operações. Ela é especialmente útil quando o número de inserções e buscas é alto e precisamos de uma estrutura de dados balanceada.
  
\end{itemize}


\subsection {(0,2) Calcule a altura da subárvore esquerda e direita da árvore binária e AVL do
exercício anterior.}

\subsection*{Solução:}

\begin{tabularx}{\textwidth}{|X|X|X|} % 'X' ajusta el ancho de las columnas
  \hline
   & Altura subarvore esquerda & Altura subarvore direita \\ \hline
   Árvore Binária & 41 & 32 \\ \hline
   AVL & 18 & 18 \\ \hline
   Vermelho-preto & 19 & 19 \\ \hline
\end{tabularx}
\\
\\
\\
A árvore vermelha-preta tem um nível a mais que a árvore AVL. Também podemos ver que está equilibrado.
\subsection*{Fontes Consultadas}
\begin{itemize}
    \item https://chatgpt.com/
\end{itemize}




\end{document}