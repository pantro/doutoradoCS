\documentclass{article}
\usepackage{array}
\usepackage{float}
\usepackage{graphicx}
\usepackage{tikz}
\usepackage{tcolorbox}
\usepackage{amsmath}
\usepackage{listings}
\usepackage{xcolor}
\definecolor{darkgreen}{rgb}{0.0, 0.4, 0.0}

\lstdefinestyle{mypython}{
    language=Python,
    backgroundcolor=\color{lightgray!20}, % Color de fondo para el código
    keywordstyle=\color{blue},            % Color para palabras clave
    stringstyle=\color{red},              % Color para strings
    commentstyle=\color{darkgreen},           % Color para comentarios
    showstringspaces=false,               % Ocultar espacios en cadenas
    basicstyle=\ttfamily\footnotesize,    % Estilo básico del código
    breaklines=true,                      % Habilitar el salto de línea automático
    frame=single,                         % Cuadro alrededor del código
}

\renewcommand\thesubsection{\arabic{subsection}} % Hace que subsección use solo números (1, 2, 3, etc.)
\setcounter{secnumdepth}{2} % Asegura que las subsecciones se numerarán

\title{Aula 06 - Exercício prático Complexidade de problemas}
\author{Aluno: Gian Franco Joel Condori Luna}
\date{\today}

\begin{document}

\maketitle

\section*{Exercices}
\setcounter{section}{1}
\subsection {(0,5) Apresente um algoritmo para o problema que pode ser resolvido em tempo
polinomial (classe P) que vc pesquisou no exercício prático. Pode usar algoritmo
seu/de outros autores, citar a fonte! Se possível mostrar a execução do algoritmo
(print do resultado) para um pequeno dataset.}

\subsection*{Solução:}

\subsection*{Problema do caminho mais curto em grafos}

\subsubsection{Explicação do Problema}

Dado um grafo direcionado com pesos não negativos, o objetivo é 
encontrar o caminho mais curto de um nó de origem para todos os 
outros nós. O algoritmo de Dijkstra oferece uma solução eficiente 
para esse problema.

\subsubsection{Algoritmo de Dijkstra}
A seguir, apresento um pseudocódigo do algoritmo de Dijkstra:
\begin{enumerate}
    \item Inicialize todas as distâncias do nó de origem para todos 
    os outros nós como infinito, exceto a distância até a origem, 
    que será 0.
    \item Coloque todos os nós em um conjunto de nós não visitados.
    \item Selecione o nó não visitado com a menor distância e marque 
    esse nó como visitado.
    \item Para cada vizinho do nó atual, calcule a distância 
    tentativa do nó de origem até esse vizinho passando pelo nó atual. 
    Se essa distância for menor que a distância armazenada 
    anteriormente, atualize-a.
    \item Repita o processo até que todos os nós tenham sido 
    visitados ou que não haja um nó alcançável.
\end{enumerate}

\subsubsection{Código em Python:}
  \begin{lstlisting}[style=mypython]
    # Fonte: ChatGPT
    # Python 3.12
    import heapq

    def dijkstra(grafo, inicio):
        # Inicializar as distancias e a fila de prioridade
        distancias = {vertice: float('inf') for vertice in grafo}
        distancias[inicio] = 0
        fila_prioridade = [(0, inicio)]
        
        while fila_prioridade:
            distancia_atual, vertice_atual = heapq.heappop(fila_prioridade)

            # Ignorar se ja encontramos uma distancia menor
            if distancia_atual > distancias[vertice_atual]:
                continue

            # Relaxamento das arestas
            for vizinho, peso in grafo[vertice_atual].items():
                distancia = distancia_atual + peso

                # Se encontrarmos um caminho mais curto
                if distancia < distancias[vizinho]:
                    distancias[vizinho] = distancia
                    heapq.heappush(fila_prioridade, (distancia, vizinho))
        
        return distancias

    # Grafo de exemplo
    grafo = {
        'A': {'B': 1, 'C': 4},
        'B': {'A': 1, 'C': 2, 'D': 5},
        'C': {'A': 4, 'B': 2, 'D': 1},
        'D': {'B': 5, 'C': 1}
    }

    # Executar Dijkstra
    vertice_inicial = 'A'
    distancias = dijkstra(grafo, vertice_inicial)

    # Imprimir resultados
    print(f"Distancias desde {vertice_inicial}:")
    for vertice, distancia in distancias.items():
        print(f"Distancia ate {vertice}: {distancia}")

  \end{lstlisting}

\subsubsection{Execução do algoritmo}
Para o pequeno conjunto de dados representado pelo grafo de exemplo, 
obtemos as distâncias mínimas do vértice 'A' para todos os outros 
vértices:  
\begin{lstlisting}[style=mypython]
  Distancias desde A:
  Distancia ate A: 0
  Distancia ate B: 1
  Distancia ate C: 3
  Distancia ate D: 4    
\end{lstlisting}

\subsubsection{Fontes Consultadas}
\begin{itemize}
    \item Cormen, T. H., Leiserson, C. E., Rivest, R. L., \& Stein, C. (2009). Introduction to Algorithms (3rd ed.). MIT Press.
    \item ChatGPT 
\end{itemize}





\subsection {(0,5) Apresente um algoritmo para o problema que pode ser resolvido em tempo não
polinomial (classe NP) que vc pesquisou no exercício prático. Pode usar algoritmo
seu/de outros autores, citar a fonte! Se possível mostrar a execução do algoritmo
(print do resultado) para um pequeno dataset.}

\subsection*{Solução:}
\subsection*{Problema do Caixeiro Viajante (TSP - Travelling Salesman Problem)}

\subsubsection{Explicação do Problema}
Dado um conjunto de cidades e as distâncias entre cada par de 
cidades, o objetivo do TSP é encontrar o caminho de menor custo que 
visita todas as cidades uma vez e retorna à cidade de origem. 
A solução por força bruta, que tenta todas as possíveis permutações 
de cidades, tem complexidade fatorial (O(n!)), onde $n$ é o número de 
cidades.

\subsubsection{Algoritmo por Força Bruta}
Um algoritmo de força bruta gera todas as permutações possíveis das 
cidades e calcula o custo de cada caminho. O caminho com o menor 
custo será a solução.

\subsubsection{Código em Python (força bruta)}
\begin{lstlisting}[style=mypython]
    # Fonte: ChatGPT
    # Python: 3.12
    from itertools import permutations

    # Funcao para calcular o custo total de um caminho
    def calcular_custo(caminho, matriz_distancias):
        custo = 0
        for i in range(len(caminho) - 1):
            custo += matriz_distancias[caminho[i]][caminho[i+1]]
        # Adicionar a volta a cidade de origem
        custo += matriz_distancias[caminho[-1]][caminho[0]]
        return custo
    
    # Algoritmo de forca bruta para resolver o TSP
    def tsp_forca_bruta(matriz_distancias):
        n = len(matriz_distancias)
        cidades = list(range(n))
        
        # Gerar todas as permutacoes das cidades
        caminhos_possiveis = permutations(cidades)
        
        # Inicializar o menor custo com infinito
        menor_custo = float('inf')
        melhor_caminho = None
        
        # Avaliar todos os caminhos possiveis
        for caminho in caminhos_possiveis:
            custo_atual = calcular_custo(caminho, matriz_distancias)
            if custo_atual < menor_custo:
                menor_custo = custo_atual
                melhor_caminho = caminho
        
        return melhor_caminho, menor_custo
    
    # Exemplo de matriz de distancias entre 4 cidades
    matriz_distancias = [
        [0, 10, 15, 20],
        [10, 0, 35, 25],
        [15, 35, 0, 30],
        [20, 25, 30, 0]
    ]
    
    # Executar o algoritmo de forca bruta
    melhor_caminho, menor_custo = tsp_forca_bruta(matriz_distancias)
    
    # Imprimir os resultados
    print(f"Melhor caminho: {melhor_caminho}")
    print(f"Menor custo: {menor_custo}")
    
\end{lstlisting}

\subsubsection{Execução do algoritmo}
Para o pequeno conjunto de dados representado por 4 cidades, o 
algoritmo gera todas as permutações de cidades e encontra o caminho 
de menor custo. A saída pode ser:

\begin{lstlisting}[style=mypython]
    Melhor caminho: (0, 1, 3, 2)
    Menor custo: 80
\end{lstlisting}

Este resultado significa que o caminho mais curto para visitar 
todas as cidades, partindo da cidade 0, passando por todas as 
outras e voltando à cidade de origem, tem um custo total de 80 
unidades.

\subsubsection{Fontes Consultadas}
\begin{itemize}
    \item Cormen, T. H., Leiserson, C. E., Rivest, R. L., \& Stein, C. (2009). Introduction to Algorithms (3rd ed.). MIT Press.
    \item ChatGPT
\end{itemize}

\end{document}