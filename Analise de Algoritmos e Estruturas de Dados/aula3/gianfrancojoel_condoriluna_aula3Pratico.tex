\documentclass{article}
\usepackage{array}
\usepackage{float}
\usepackage{graphicx}
\usepackage{listings}
\usepackage{xcolor}
\usepackage{hyperref}

\lstset{
    language=Python, % especifica el lenguaje de programación
    basicstyle=\ttfamily, % estilo básico del texto
    keywordstyle=\color{blue}, % color de las palabras clave
    commentstyle=\color{gray}, % color de los comentarios
    stringstyle=\color{red}, % color de las cadenas
    showstringspaces=false % no mostrar espacios en cadenas
}

\title{Aula 03 - Exercício prático Ordenação Quadrática}
\author{Aluno: Gian Franco Joel Condori Luna}
\date{\today}

\begin{document}

\maketitle

\section*{Exercício no Beecrowd}

Para resolver o problema de permutações ordenadas no ambiente Beecrowd contamos com o algoritmo Trotter-Johnson.
\\
\\
Referências: 
\begin{itemize}
    \item \href{https://theswissbay.ch/pdf/Gentoomen%20Library/Algorithms/Algorithms%20in%20C.pdf}{Algorithms in C}.
    \item \href{https://www3.decom.ufop.br/toffolo/site_media/uploads/2011-1/bcc402/slides/07._recursao.pdf}{Slides decom.ufop}.
\end{itemize}

\setcounter{section}{01}

\subsection {Algoritmo Trotter-Johnson}

O algoritmo Trotter-Johnson, também conhecido como algoritmo de permutação Steinhaus-Johnson-Trotter ou 
algoritmo de permutação de transposição, foi desenvolvido independentemente por várias pessoas. O algoritmo 
foi popularizado na literatura matemática e é amplamente utilizado devido à sua simplicidade e eficiência.


Este algoritmo é conhecido por sua capacidade de gerar permutações sucessivas através de transposições 
simples (trocas). O algoritmo segue estas etapas gerais:

\begin{itemize}
    \item Caso base: Se o conjunto tiver apenas um elemento, a permutação é o próprio conjunto.

    \item Recursão: Para gerar as permutações de um conjunto de n elementos:
    \begin{itemize}
        \item Primeiro, gere todas as permutações do subconjunto dos primeiros n-1 elementos.
        \item Em seguida, insere o enésimo elemento em todas as posições possíveis das permutações geradas acima.
    \end{itemize}
    
    \item Inserções alternadas: A direção em que o enésimo elemento é inserido alterna da esquerda para a direita e da direita para a esquerda para evitar duplicatas e garantir que cada permutação difere da anterior em apenas uma troca.
\end{itemize}

Aqui mostramos o pseudocódigo que descreve o algoritmo Trotter – Johnson, que é a base para o código feito em Python:

\begin{lstlisting}
procedure generate_permutations(n):
    p := array of integers from 1 to n
    c := array of zeros with length n
    print p

    i := 0
    while i < n:
        if c[i] < i:
            if i is even then
                swap p[0] and p[i]
            else
                swap p[c[i]] and p[i]

            print p
            c[i] += 1
            i := 0
        else
            c[i] := 0
            i += 1
\end{lstlisting}

\subsection {Análise de complexidade}

\begin{lstlisting}[breaklines=true]
def trotter_johnson_permutations(lista):
  result = []
  n = len(lista)
  c = [0] * n  # Inicialize os contadores
  result.append(''.join(lista))
    
  i = 0
  while i < n:
    if c[i] < i:
      if i % 2 == 0:
        swap_with = 0
      else:
        swap_with = c[i]
                
      # Troque os itens
      lista[swap_with], lista[i] = lista[i], lista[swap_with]
      result.append(''.join(lista))
            
      c[i] += 1
      i = 0
    else:
      c[i] = 0
      i += 1
  return result
//Fim da funcao trotter_johnson_permutations

n = int(input())
for _ in range(n):
  cadena = input()
  lista = list(cadena)
  result = trotter_johnson_permutations(lista)
  result = list(set(result))
  result.sort()
    
  for r in result:
    print(r)
  print()
\end{lstlisting}

\subsubsection {Inicialización:}
\begin{itemize}
    \item $result = [ ]$: Operação constante $ O(1) $
    \item $n = len(lista)$: Operação constante $ O(1) $
    \item $c = [0] * n$  : Operação linear $ O(n) $
\end{itemize}

\subsubsection{Loop principal ($ while \quad i < n $):}

Tudo dentro do $ while $ será executado $ n $ vezes

\begin{itemize}
    \item $lista[swap\_with], lista[i] = lista[i], lista[swap\_with]$ : Operação fatorial $ O(n!)$ 
    pois terá que realizar todas as permutações que existem de $ n $
\end{itemize}
Portanto o while e o que ele contém (permutações) serão executados $O(n!*n)$

\subsection {Resposta}

\[ O(n)  + O(n!*n) = O(n!*n)\]

\end{document}