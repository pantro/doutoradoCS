\documentclass{article}
\usepackage[table]{xcolor}
\usepackage{array}
\usepackage{float}
\usepackage{graphicx}
\usepackage{tikz}
\usepackage{tcolorbox}
\usepackage{amsmath}
\usepackage{enumitem}

\renewcommand\thesubsection{\arabic{subsection}} % Hace que subsección use solo números (1, 2, 3, etc.)
\setcounter{secnumdepth}{2} % Asegura que las subsecciones se numerarán

\title{Aula 07 - Exercício teórico Variações de listas}
\author{Aluno: Gian Franco Joel Condori Luna}
\date{\today}

\begin{document}

\maketitle

\section*{Exercices}
\setcounter{section}{1}
\subsection {(1,0) Descreva outras variações existentes sobre listas e se elas apresentam alguma
vantagem em determinada aplicação:}
\begin{enumerate}[label=\alph*)]
    \item Listas circulares
    \item Listas duplamente encadeadas
    \item Tercer Outra variação que vc tenha visto
\end{enumerate}

\subsection*{Solução:}

\subsection*{a) Listas Circulares}

As listas circulares são uma variação das listas encadeadas, onde o 
último nó da lista aponta para o primeiro nó, formando assim um ciclo 
fechado. Dependendo da implementação, podem ser simplesmente encadeadas 
(com ponteiros apenas para o próximo elemento) ou duplamente encadeadas 
(com ponteiros para o próximo e o anterior).

\subsubsection{Características:}
\begin{itemize}
    \item Circularidade: O último nó aponta para o primeiro, permitindo que, ao percorrer a lista, o processo se torne infinito sem a necessidade de verificações adicionais.
    \item Ausência de nó nulo: Não existe um ponteiro que aponte para NULL como nas listas encadeadas tradicionais, tornando-a contínua e ideal para situações que envolvem repetição contínua de elementos.
    \item Eficiência: Operações como adição e remoção de nós nas extremidades podem ser feitas em tempo constante (O(1)), especialmente se um ponteiro para o final da lista for mantido.
    
\end{itemize}

\subsubsection{Vantagens:}
\begin{itemize}
    \item Ideal para navegação contínua: Listas circulares são especialmente úteis em aplicações que exigem repetição ou ciclos contínuos de elementos, como em sistemas que utilizam filas de processos ou buffers de dados.
    \item Uso eficiente de memória: Por não precisar de ponteiros nulos ou verificações de final de lista, pode ser mais eficiente em termos de memória e processamento.
\end{itemize}

\subsubsection{Aplicações:}
\begin{itemize}
    \item Escalonadores de processos: Em sistemas operacionais que utilizam o algoritmo de escalonamento round-robin, onde cada processo recebe um tempo igual de execução, uma lista circular garante que, ao fim da execução de um processo, o sistema automaticamente passe ao próximo sem precisar reiniciar a lista manualmente.

    Exemplo: Em um escalonador de processos, cada nó da lista circular pode representar um processo. Após a execução de um processo, o sistema segue automaticamente para o próximo processo na lista, e quando o último processo é executado, o ciclo recomeça no primeiro processo.
    
    \item Jogos Multiplayer: Em jogos baseados em turnos, como jogos de tabuleiro ou cartas, listas circulares podem ser usadas para gerenciar a ordem dos jogadores, garantindo que, após o último jogador, o jogo volte ao primeiro.
    
    \item Buffers Circulares (Ring Buffers): Utilizados em sistemas que precisam de armazenamento contínuo de dados, como streams de áudio e vídeo, onde novos dados sobrescrevem os mais antigos à medida que o buffer se preenche.
\end{itemize}

\subsection*{b) Listas Duplamente Encadeadas}

As listas duplamente encadeadas, ou duplamente ligadas, possuem nós que contêm dois ponteiros: um apontando para o próximo nó e outro para o nó anterior. Isso as diferencia das listas encadeadas simples, que permitem navegação apenas em uma direção (do primeiro para o último nó).

\subsubsection{Características:}
\begin{itemize}
    \item Navegação bidirecional: Cada nó tem um ponteiro que permite tanto avançar quanto recuar na lista, facilitando operações que precisam de acesso em ambas direções.
    \item Maior memória por nó: Devido aos dois ponteiros (para o anterior e o próximo), cada nó de uma lista duplamente encadeada requer mais memória do que um nó de uma lista encadeada simples.
    \item Facilidade para operações de remoção: Em listas duplamente encadeadas, a remoção de um nó é mais eficiente, pois há acesso direto ao nó anterior e ao próximo, sem necessidade de percorrer a lista inteira.
    
\end{itemize}

\subsubsection{Vantagens:}
\begin{itemize}
    \item Eficiência na remoção e inserção: Quando se conhece a posição do nó a ser removido ou inserido, as operações são realizadas de forma muito eficiente em tempo constante (O(1)).
    \item Navegação em ambas direções: É possível mover-se para frente e para trás facilmente, o que é útil em situações onde a navegação reversa é frequente, como em editores de texto ou sistemas de navegação de páginas.  

\end{itemize}

\subsubsection{Aplicações:}
\begin{itemize}
    \item Navegadores Web: Muitos navegadores web usam listas duplamente encadeadas para gerenciar o histórico de navegação. Quando o usuário clica no botão "voltar", o navegador acessa o nó anterior na lista (a página anterior), e ao clicar em "avançar", ele acessa o próximo nó (a página seguinte).

    \item Editor de Texto: Em um editor de texto, cada caractere ou linha pode ser representado por um nó em uma lista duplamente encadeada. O cursor pode mover-se tanto para a esquerda quanto para a direita, facilitando a inserção, remoção e modificação de texto de forma eficiente.
    
    \item Gerenciamento de memória: Em alguns sistemas de gerenciamento de memória, listas duplamente encadeadas são usadas para manter uma lista de blocos de memória livres e ocupados. Isso permite a fusão rápida de blocos adjacentes de memória durante a liberação de memória.
    
    \item Estruturas de dados como Deques (Double-ended Queues): Deques permitem inserção e remoção de elementos tanto no início quanto no final da lista, e são frequentemente implementados usando listas duplamente encadeadas para garantir essa flexibilidade.
\end{itemize}

\subsection*{c) Outra variação: Listas Autorreferenciadas}

Uma variação interessante são as listas autoajustáveis ou listas com acesso autoajustável. Nesses tipos de lista, a estrutura da lista é modificada com base nos padrões de acesso. Um exemplo clássico é a lista de transposição, onde um elemento acessado é movido para a frente da lista, de modo que os elementos mais frequentemente acessados fiquem mais próximos do início.

\subsubsection{Características:}
\begin{itemize}
    \item Autoajuste dinâmico: A lista se ajusta dinamicamente com base 
    na fre-quência de acessos. Elementos acessados frequentemente são 
    movidos para o início da lista, melhorando o tempo de acesso para 
    esses elementos.
    \item Implementação flexível: Pode ser implementada tanto em listas 
    encadea-das simples quanto em listas duplamente encadeadas, 
    dependendo da necessidade de navegação.
\end{itemize}

\subsubsection{Vantagens:}
\begin{itemize}
    \item Melhora o tempo de acesso: Ao mover elementos frequentemente acessados para o início da lista, o tempo de acesso para esses elementos melhora consideravelmente em cenários onde certos elementos são usados mais do que outros.
    \item Eficiência sem estrutura adicional: Não requer uma estrutura complexa como uma árvore balanceada ou tabela hash, mas ainda assim oferece um ganho significativo de desempenho em certos cenários de acesso.
\end{itemize}

\subsubsection{Aplicações:}
\begin{itemize}
    \item Sistemas de Cache: Listas autoajustáveis podem ser usadas em sistemas de cache, onde os dados mais acessados recentemente são movidos para o início da lista, melhorando o tempo de acesso futuro.

    \item Compressão de dados: Algoritmos de compressão como o Move-To-Front (MTF) usam listas autoajustáveis para reordenar símbolos com base na frequência de uso, otimizando a compressão de dados.
\end{itemize}

\subsubsection{Fontes Consultadas}
\begin{itemize}
    \item https://chatgpt.com/
\end{itemize}




\end{document}