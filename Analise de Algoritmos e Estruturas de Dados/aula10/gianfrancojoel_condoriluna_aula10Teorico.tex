\documentclass{article}
\usepackage{enumitem}
\usepackage{amsmath}
\usepackage{tikz}
\usepackage[utf8]{inputenc}   % Suporte para codificação UTF-8
\usetikzlibrary{positioning}

\renewcommand\thesubsection{\arabic{subsection}} % Hace que subsección use solo números (1, 2, 3, etc.)
\setcounter{secnumdepth}{2} % Asegura que las subsecciones se numerarán

\title{AULA 10: Exercício teórico Tabelas Hash}
\author{Aluno: Gian Franco Joel Condori Luna}
\date{\today}

\begin{document}

\maketitle

\section*{Exercices}
\setcounter{section}{1}
\subsection {(0,2) Mostre a inserção das chaves 5, 28, 19, 15, 20, 33, 12, 17, 10 em uma tabela hash
com colisões resolvidas por encadeamento. Seja a tabela com 9 posições, e seja a
função hash h(k) = k mod 9.}

\subsubsection{Solução:}

- \textbf{Posições:} \( m = 9 \).\\
- \textbf{Função hash:} \( h(k) = k \mod 9 \).\\
- \textbf{Chaves:} \(  5, 28, 19, 15, 20, 33, 12, 17, 10 \).

\begin{center}
\begin{tabular}{|c|c|}
\hline
\textbf{Índice} & \textbf{Chaves} \\
\hline
0 &  \\
1 &  \\
2 &  \\
3 &  \\
4 &  \\
5 &  \\
6 &  \\
7 &  \\
8 &  \\
\hline
\end{tabular}
\end{center}

\begin{enumerate}
  \item Inserindo 5:
    \begin{itemize}
      \item \(5 \rightarrow h(5) = 5 \mod 9 = 5\)
          
      \begin{center}
        \begin{tabular}{|c|c|}
        \hline
        \textbf{Índice} & \textbf{Chaves} \\
        \hline
        0 &  \\
        1 &  \\
        2 & \\
        3 & \\
        4 &  \\
        5 & 5 \\
        6 & \\
        7 &  \\
        8 & \\
        \hline
        \end{tabular}
      \end{center}
    \end{itemize}
  \item Inserindo 28:
    \begin{itemize}
      \item \(28 \rightarrow h(28) = 28 \mod 9 = 1\)
          
      \begin{center}
        \begin{tabular}{|c|c|}
        \hline
        \textbf{Índice} & \textbf{Chaves} \\
        \hline
        0 &  \\
        1 & 28 \\
        2 & \\
        3 & \\
        4 &  \\
        5 & 5 \\
        6 & \\
        7 &  \\
        8 & \\
        \hline
        \end{tabular}
      \end{center}
    \end{itemize}
  \item Inserindo 19:
    \begin{itemize}
      \item \(19 \rightarrow h(19) = 19 \mod 9 = 1\) (colisão com 28)
          
      \begin{center}
        \begin{tabular}{|c|c|}
        \hline
        \textbf{Índice} & \textbf{Chaves} \\
        \hline
        0 &  \\
        1 & 28 $\rightarrow$ 19\\
        2 & \\
        3 & \\
        4 &  \\
        5 & 5 \\
        6 & \\
        7 &  \\
        8 & \\
        \hline
        \end{tabular}
      \end{center}
    \end{itemize}
  \item Inserindo 15:
    \begin{itemize}
      \item \(15 \rightarrow h(15) = 15 \mod 9 = 6\)
          
      \begin{center}
        \begin{tabular}{|c|c|}
        \hline
        \textbf{Índice} & \textbf{Chaves} \\
        \hline
        0 &  \\
        1 & 28 $\rightarrow$ 19\\
        2 & \\
        3 & \\
        4 &  \\
        5 & 5 \\
        6 & 15\\
        7 &  \\
        8 & \\
        \hline
        \end{tabular}
      \end{center}
    \end{itemize}
  \item Inserindo 20:
    \begin{itemize}
      \item \(20 \rightarrow h(20) = 20 \mod 9 = 2\)
          
      \begin{center}
        \begin{tabular}{|c|c|}
        \hline
        \textbf{Índice} & \textbf{Chaves} \\
        \hline
        0 &  \\
        1 & 28 $\rightarrow$ 19\\
        2 & 20\\
        3 & \\
        4 &  \\
        5 & 5 \\
        6 & 15\\
        7 &  \\
        8 & \\
        \hline
        \end{tabular}
      \end{center}
    \end{itemize}
  \item Inserindo 33:
    \begin{itemize}
      \item \(33 \rightarrow h(33) = 33 \mod 9 = 6\) (colisão com 15)
          
      \begin{center}
        \begin{tabular}{|c|c|}
        \hline
        \textbf{Índice} & \textbf{Chaves} \\
        \hline
        0 &  \\
        1 & 28 $\rightarrow$ 19\\
        2 & 20\\
        3 & \\
        4 &  \\
        5 & 5 \\
        6 & 15 $\rightarrow$ 33\\
        7 &  \\
        8 & \\
        \hline
        \end{tabular}
      \end{center}
    \end{itemize}
  \item Inserindo 12:
    \begin{itemize}
      \item \(12 \rightarrow h(12) = 12 \mod 9 = 3\)
          
      \begin{center}
        \begin{tabular}{|c|c|}
        \hline
        \textbf{Índice} & \textbf{Chaves} \\
        \hline
        0 &  \\
        1 & 28 $\rightarrow$ 19\\
        2 & 20\\
        3 & 12\\
        4 &  \\
        5 & 5 \\
        6 & 15 $\rightarrow$ 33\\
        7 &  \\
        8 & \\
        \hline
        \end{tabular}
      \end{center}
    \end{itemize}
  \item Inserindo 17:
    \begin{itemize}
      \item \(17 \rightarrow h(17) = 17 \mod 9 = 8\)
          
      \begin{center}
        \begin{tabular}{|c|c|}
        \hline
        \textbf{Índice} & \textbf{Chaves} \\
        \hline
        0 &  \\
        1 & 28 $\rightarrow$ 19\\
        2 & 20\\
        3 & 12\\
        4 &  \\
        5 & 5 \\
        6 & 15 $\rightarrow$ 33\\
        7 &  \\
        8 & 17\\
        \hline
        \end{tabular}
      \end{center}
    \end{itemize}
  \item Inserindo 10:
    \begin{itemize}
      \item \(10 \rightarrow h(10) = 10 \mod 9 = 1\) (colisão com 28 e 19)
          
      \begin{center}
        \begin{tabular}{|c|c|}
        \hline
        \textbf{Índice} & \textbf{Chaves} \\
        \hline
        0 &  \\
        1 & 28 $\rightarrow$ 19 $\rightarrow$ 10\\
        2 & 20\\
        3 & 12\\
        4 &  \\
        5 & 5 \\
        6 & 15 $\rightarrow$ 33\\
        7 &  \\
        8 & 17\\
        \hline
        \end{tabular}
      \end{center}
    \end{itemize}
\end{enumerate}

\subsection {(0,2) (0,2) Considere uma tabela hash de tamanho m = 1000 e a função hash
correspondente h(k) igual a $
   h(k) = \lfloor m \cdot (k \cdot A \mod 1) \rfloor
   $
para A = $ \quad A = \frac{\sqrt{5} - 1}{2}$.
Calcule as localizações para as quais as chaves 61, 62, 63, 64 e 35 estão mapeadas.}

\subsubsection{Solução:}

- \textbf{Tamanho:} $ m = 1000$ \\
- \textbf{h(k) = } $ \lfloor m \cdot (k \cdot A \mod 1) \rfloor
   $\\
- \textbf{A = } $ \frac{\sqrt{5} - 1}{2} \approx 0.6180339887
   $

   \begin{itemize}
       \item Calculando para \( k = 61 \):
       \[
       h(61) = \lfloor 1000 \cdot (61 \cdot A \mod 1) \rfloor = \lfloor 699.0733 \rfloor = 699
       \]

       \item Calculando para \( k = 62 \):
       \[
       h(62) = \lfloor 1000 \cdot (62 \cdot A \mod 1) \rfloor = \lfloor 317.1073 \rfloor = 317
       \]

       \item Calculando para \( k = 63 \):
       \[
       h(63) = \lfloor 1000 \cdot (63 \cdot A \mod 1) \rfloor = \lfloor 935.1413 \rfloor = 935
       \]

       \item Calculando para \( k = 64 \):
       \[
       h(64) = \lfloor 1000 \cdot (64 \cdot A \mod 1) \rfloor = \lfloor 553.1752 \rfloor = 553
       \]

       \item Calculando para \( k = 35 \):
       \[
       h(35) = \lfloor 1000 \cdot (35 \cdot A \mod 1) \rfloor = \lfloor 631.1896 \rfloor = 631
       \]
   \end{itemize}

\subsection {(0,6) Considere a inserção das chaves 10, 22, 31, 4, 15, 28, 17, 88, 59 em uma tabela
hash de comprimento m = 11 usando o endereçamento aberto com a função hash
primário h(k) = k mod m.
Ilustre o resultado da inserção dessas chaves com:}
\begin{enumerate}[label=\alph*)]
  \item O uso da sondagem linear
  \item O uso da sondagem quadrática com c1 = 1 e c2 = 3
  \item O uso do hash duplo com $ h2(k) = 1 + (k \mod (m - 1))$
\end{enumerate}

\subsubsection{Solução:}

\begin{enumerate}[label=\alph*)]
  \item \textbf{Inserção com Sondagem Linear}
  \\\\
    Função de tentativa é dada por:
    \[
    h_i(k) = (h(k) + i) \mod 11
    \]
    onde \(i\) representa o número de colisões já ocorridas.

    Na tabela está o processo de inserção das chaves com a sondagem linear:

    \begin{center}
    \begin{tabular}{|c|c|c|c|}
    \hline
    Chave \(k\) & \(h(k)\) & Tentativas & Posição Final \\
    \hline
    10 & \(10 \mod 11 = 10\) & 0 & 10 \\
    22 & \(22 \mod 11 = 0\) & 0 & 0 \\
    31 & \(31 \mod 11 = 9\) & 0 & 9 \\
    4 & \(4 \mod 11 = 4\) & 0 & 4 \\
    15 & \(15 \mod 11 = 4\) & 1 & 5 \\
    28 & \(28 \mod 11 = 6\) & 0 & 6 \\
    17 & \(17 \mod 11 = 6\) & 1 & 7 \\
    88 & \(88 \mod 11 = 0\) & 1 & 1 \\
    59 & \(59 \mod 11 = 4\) & 2 & 8 \\
    \hline
    \end{tabular}
    \end{center}

  \item \textbf{Inserção com Sondagem Quadrática}
  \\\\
  Para a sondagem quadrática, usamos a função:
  \[
  h_i(k) = (h(k) + c_1 \cdot i + c_2 \cdot i^2) \mod 11
  \]
  com \(c_1 = 1\) e \(c_2 = 3\).
  
  \begin{center}
  \begin{tabular}{|c|c|c|c|}
  \hline
  Chave \(k\) & \(h(k)\) & Tentativas & Posição Final \\
  \hline
  10 & 10 & 0 & 10 \\
  22 & 0 & 0 & 0 \\
  31 & 9 & 0 & 9 \\
  4 & 4 & 0 & 4 \\
  15 & 4 & 1 & 8 \\
  28 & 6 & 0 & 6 \\
  17 & 6 & 1 & 10 \\
  88 & 0 & 1 & 4 (colisão) \\
  88 & 0 & 2 & 0 (colisão) \\
  88 & 0 & 3 & 5 \\
  59 & 4 & 1 & 8 (colisão) \\
  59 & 4 & 2 & 1 \\
  \hline
  \end{tabular}
  \end{center}
  
  \item \textbf{Inserção com Hash Duplo}
  \\\\
    Para o hash duplo, a função é:
    \[
    h_i(k) = (h(k) + i \cdot h_2(k)) \mod 11
    \]
    onde \(h_2(k) = 1 + (k \mod (m - 1))\).

    \begin{center}
    \begin{tabular}{|c|c|c|c|c|}
    \hline
    Chave \(k\) & \(h(k)\) & \(h_2(k)\) & Tentativas & Posição Final \\
    \hline
    10 & 10 & 2 & 0 & 10 \\
    22 & 0 & 4 & 0 & 0 \\
    31 & 9 & 5 & 0 & 9 \\
    4 & 4 & 5 & 0 & 4 \\
    15 & 4 & 8 & 1 & 1 \\
    28 & 6 & 3 & 0 & 6 \\
    17 & 6 & 10 & 1 & 7 \\
    88 & 0 & 1 & 1 & 1 (colisão) \\
    88 & 0 & 1 & 2 & 2 \\
    59 & 4 & 6 & 1 & 8 \\
    \hline
    \end{tabular}
    \end{center}
\end{enumerate}

\end{document}