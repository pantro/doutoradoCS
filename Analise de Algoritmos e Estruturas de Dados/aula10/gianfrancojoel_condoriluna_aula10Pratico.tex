\documentclass{article}
\usepackage[table]{xcolor}
\usepackage{array}
\usepackage{float}
\usepackage{graphicx}
\usepackage{tikz}
\usepackage{tcolorbox}
\usepackage{amsmath}
\usepackage{enumitem}
\usepackage{tabularx} % Para tablas que se ajustan al ancho de la página
\usepackage{booktabs} % Para líneas mejoradas en la tabla

\renewcommand\thesubsection{\arabic{subsection}} % Hace que subsección use solo números (1, 2, 3, etc.)
\setcounter{secnumdepth}{2} % Asegura que las subsecciones se numerarán

\title{Aula 10 - Exercício prático Tabela Hash}
\author{Aluno: Gian Franco Joel Condori Luna}
\date{\today}

\begin{document}

\maketitle

\section*{Exercices}
\setcounter{section}{1}
\subsection {(1,0) Implemente um algoritmo para função Hash que:}
\begin{enumerate}[label=\alph*)]
    \item Faz resolução de colisões por encadeamento com a função hash da divisão.
    \item Faz resolução de colisões por endereçamento aberto usando sondagem linear.
    \item Insira 100.000 elementos gerados aleatoriamente no intervalo [0-100.000] em uma
    tabela hash com m = 50.000 posições no encadeamento e m = 100.000 no
    endereçamento aberto e compute o tempo de inserção em cada caso. Depois pesquise
    o elemento 10.000:
\end{enumerate}

\subsection*{Solução:}

(O código está no arquivo python)
\\

\begin{tabularx}{\textwidth}{|>{\centering\arraybackslash}X|>{\centering\arraybackslash}X|} 
  \hline
   \textbf{Hash por encadeamento} & 
   \textbf{Hash por endereçamento aberto} \\ 
  \hline
  Tempo Inserção Função hash divisão: 0.0372 s  
  & Tempo Inserção Sondagem Linear: 2.1680 s  
  \\ \hline
  Tempo Busca Função hash divisão: 0.000038 s  
  & Tempo Busca Sondagem Linear: 0.000032 s  
  \\ \hline
\end{tabularx}
\\
\\Discussão:
\begin{itemize}
  \item \textbf{Tempo de Inserção:} A função hash com encadeamento levou significativamente menos tempo para inserir os elementos (0,0372 s) comparado ao método de endereçamento aberto com sondagem linear (2,1680 s).
  Isso sugere que o encadeamento é mais eficiente para inserções quando a tabela hash tem um número elevado de elementos e há alta possibilidade de colisões.

  \item \textbf{Tempo de Busca:} Para a busca, ambos os métodos apresentaram tempos de execução muito pequenos, com uma diferença mínima entre eles (0,000038 s para encadeamento e 0,000032 s para sondagem linear).
  Embora o tempo de busca seja similar, isso indica que ambos os métodos são eficientes para buscas rápidas, mas o encadeamento ainda pode ser uma escolha mais vantajosa para um número elevado de inserções.
  
\end{itemize}


\subsection*{Fontes Consultadas}
\begin{itemize}
    \item https://chatgpt.com/
\end{itemize}




\end{document}